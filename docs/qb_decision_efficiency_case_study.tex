% Options for packages loaded elsewhere
\PassOptionsToPackage{unicode}{hyperref}
\PassOptionsToPackage{hyphens}{url}
\documentclass[
  16pt,
]{article}
\usepackage{xcolor}
\usepackage[margin=1in]{geometry}
\usepackage{amsmath,amssymb}
\setcounter{secnumdepth}{-\maxdimen} % remove section numbering
\usepackage{iftex}
\ifPDFTeX
  \usepackage[T1]{fontenc}
  \usepackage[utf8]{inputenc}
  \usepackage{textcomp} % provide euro and other symbols
\else % if luatex or xetex
  \usepackage{unicode-math} % this also loads fontspec
  \defaultfontfeatures{Scale=MatchLowercase}
  \defaultfontfeatures[\rmfamily]{Ligatures=TeX,Scale=1}
\fi
\usepackage{lmodern}
\ifPDFTeX\else
  % xetex/luatex font selection
  \setmainfont[]{Verdana}
\fi
% Use upquote if available, for straight quotes in verbatim environments
\IfFileExists{upquote.sty}{\usepackage{upquote}}{}
\IfFileExists{microtype.sty}{% use microtype if available
  \usepackage[]{microtype}
  \UseMicrotypeSet[protrusion]{basicmath} % disable protrusion for tt fonts
}{}
\makeatletter
\@ifundefined{KOMAClassName}{% if non-KOMA class
  \IfFileExists{parskip.sty}{%
    \usepackage{parskip}
  }{% else
    \setlength{\parindent}{0pt}
    \setlength{\parskip}{6pt plus 2pt minus 1pt}}
}{% if KOMA class
  \KOMAoptions{parskip=half}}
\makeatother
\usepackage{graphicx}
\makeatletter
\newsavebox\pandoc@box
\newcommand*\pandocbounded[1]{% scales image to fit in text height/width
  \sbox\pandoc@box{#1}%
  \Gscale@div\@tempa{\textheight}{\dimexpr\ht\pandoc@box+\dp\pandoc@box\relax}%
  \Gscale@div\@tempb{\linewidth}{\wd\pandoc@box}%
  \ifdim\@tempb\p@<\@tempa\p@\let\@tempa\@tempb\fi% select the smaller of both
  \ifdim\@tempa\p@<\p@\scalebox{\@tempa}{\usebox\pandoc@box}%
  \else\usebox{\pandoc@box}%
  \fi%
}
% Set default figure placement to htbp
\def\fps@figure{htbp}
\makeatother
\setlength{\emergencystretch}{3em} % prevent overfull lines
\providecommand{\tightlist}{%
  \setlength{\itemsep}{0pt}\setlength{\parskip}{0pt}}
\usepackage{bookmark}
\IfFileExists{xurl.sty}{\usepackage{xurl}}{} % add URL line breaks if available
\urlstyle{same}
\hypersetup{
  pdftitle={Quarterback Decision Efficiency Case Study (2021--2024)},
  pdfauthor={Drake Soltez},
  hidelinks,
  pdfcreator={LaTeX via pandoc}}

\title{Quarterback Decision Efficiency Case Study (2021--2024)}
\author{Drake Soltez}
\date{}

\begin{document}
\maketitle

{
\setcounter{tocdepth}{2}
\tableofcontents
}
\section{1. Ask}\label{ask}

\subsection{1.1 Business Problem}\label{business-problem}

The goal of this case study is to understand how quarterback
decision-making impacts offensive efficiency and drive success across
the 2021--2024 NFL seasons. This project evaluates how often
quarterbacks make effective decisions and how these decisions relate to
performance, consistency, and tempo.

\subsection{1.2 Key Questions}\label{key-questions}

\begin{enumerate}
\def\labelenumi{\arabic{enumi}.}
\tightlist
\item
  Which quarterbacks had the highest decision rate (percentage of
  effective decisions per dropback)?\\
\item
  Which quarterbacks make consistently high-impact decisions (DEI)?\\
\item
  Which quarterbacks show improvement or decline in decision efficiency
  over the four seasons?\\
\item
  Is there a relationship between decision effectiveness and decision
  tempo (time to throw)?
\end{enumerate}

\subsection{1.3 Stakeholders}\label{stakeholders}

\begin{itemize}
\tightlist
\item
  NFL analytics departments\\
\item
  Coaching staff and coordinators\\
\item
  Player personnel and scouting\\
\item
  Media analysts and endorsement strategists
\end{itemize}

\section{2. Prepare}\label{prepare}

\subsection{2.1 Data Sources}\label{data-sources}

This analysis uses:

\begin{itemize}
\tightlist
\item
  NFL play-by-play data (2021--2024)\\
\item
  Next Gen Stats passing data\\
\item
  Additional QB-level summaries created during processing
\end{itemize}

All data was accessed through the \textbf{nflreadr} and
\textbf{nflverse} R packages.

\subsection{2.2 Data Filters Applied}\label{data-filters-applied}

\begin{itemize}
\tightlist
\item
  Only quarterback dropbacks included\\
\item
  Minimum 300 dropbacks per season for multi-season comparability
\end{itemize}

Built:\\
- Season-level QB summary tables\\
- 4-season combined summaries\\
- DEI rankings (full 4-season QBs + minimum 1-season QB)

\subsection{2.3 Data Cleaning}\label{data-cleaning}

Before analysis, extensive data cleaning was required to ensure accuracy
and consistency across all four seasons of play-by-play and Next Gen
Stats data. Key cleaning steps included:

\paragraph{Play-by-Play Cleaning}\label{play-by-play-cleaning}

\begin{itemize}
\item
  Filtered to quarterback dropbacks only, removing all non-QB plays to
  ensure decision-making metrics were measured only on plays where the
  quarterback was the primary decision-maker.
\item
  Excluded all designed run plays (RB handoffs, end-arounds, sweeps) and
  kept only plays where a dropback occurred (pass attempts, sacks,
  scrambles, throwaways).
\item
  Removed QB scrambles that had no passer associated, since they do not
  contain valid passing decision data.
\item
  Standardized QB names across seasons to avoid mismatches caused by
  abbreviations, suffixes (e.g., Jr.), or inconsistent formats in the
  source data.
\end{itemize}

\paragraph{Next Gen Stats (NGS)
Cleaning}\label{next-gen-stats-ngs-cleaning}

\begin{itemize}
\item
  Filtered NGS passing data to quarterbacks only, eliminating non-QB
  positional anomalies that sometimes appear in the raw files.
\item
  Selected and cleaned key NGS variables such as:
\item
  avg\_time\_to\_throw (decision tempo)
\item
  QB identifiers (player\_short\_name, player\_display\_name)
\item
  pass\_touchdowns and other relevant passing metrics
\item
  Normalized player names in NGS data to match those in PBP data for
  accurate merging.
\end{itemize}

\paragraph{Merging \& Final
Standardization}\label{merging-final-standardization}

\begin{itemize}
\item
  Performed a left join to merge PBP summaries with NGS passing data by
  QB and season.
\item
  Validated merges to ensure no quarterbacks were duplicated or dropped.
\item
  Applied filters for 300+ dropbacks to define ``qualified''
  quarterbacks for multi-season comparison.
\item
  Removed unused columns and reordered remaining variables for clarity
  and consistent reporting.
\end{itemize}

\section{3. Process}\label{process}

\subsection{3.1 Decision Effectiveness
Definition}\label{decision-effectiveness-definition}

A play is labeled \textbf{decision-effective} if it results in:

\begin{itemize}
\tightlist
\item
  Positive EPA\\
\item
  First down or touchdown\\
\item
  No turnover\\
\item
  No blown negative outcome
\end{itemize}

We define \textbf{Decision Rate} as:

\[
\text{Decision Rate} = \frac{\text{Effective Decisions}}{\text{Dropbacks}}
\]

\subsection{3.2 QB Summary Metrics}\label{qb-summary-metrics}

For each QB-season:

\begin{itemize}
\tightlist
\item
  Dropbacks\\
\item
  Effective decisions\\
\item
  Decision rate\\
\item
  EPA per play\\
\item
  Interceptions\\
\item
  Fumbles lost\\
\item
  Turnover rate\\
\item
  Touchdowns\\
\item
  Average time to throw (tempo)
\end{itemize}

\section{4. Analyze}\label{analyze}

\subsection{4.1 - Top Decision Rate QB's}\label{top-decision-rate-qbs}

\subsection{4.1.1 Question 1 --- Which Quarterbacks had the highest
decision rate (percentage of effective decisions per dropback)
throughout the four
seasons?}\label{question-1-which-quarterbacks-had-the-highest-decision-rate-percentage-of-effective-decisions-per-dropback-throughout-the-four-seasons}

\begin{figure}

{\centering \includegraphics{qb_decision_efficiency_case_study_files/figure-latex/decision_effective_qbs-1} 

}

\caption{Top 10 Quarterbacks by Decision Rate (2021–2024)}\label{fig:decision_effective_qbs}
\end{figure}

\textbf{Summary (Q1):}\\
The vertical bar chart is the strongest choice for comparing
quarterbacks on a single percentage-based metric like \textbf{decision
rate}, because it allows viewers to immediately identify who performs
best. Sorting the bars from highest to lowest makes the top
decision-makers easy to spot, while the color scale representing
dropbacks adds important context about workload. This helps distinguish
quarterbacks who were efficient on a high volume of plays from those who
performed well on smaller samples.

\textbf{Key findings from the graph:}\\
From this dataset, the chart shows that the highest decision rates
belong to \textbf{Patrick Mahomes}, \textbf{Tua Tagovailoa}, and
\textbf{Jared Goff}, who all sustain very high rates of effective
decisions per dropback, even if the quarterback had less dropbacks than
other quarterbacks. The difference between the top few quarterbacks and
the rest of the top ten forms a clear ``top tier,'' while the remaining
QBs in the chart are more tightly clustered. This indicates that a small
group of passers consistently separate themselves in decision quality,
while most of the league's top performers operate within a narrower
efficiency band.

This visualization directly answers Question 1 by revealing which
quarterbacks most consistently avoided negative outcomes and sustained
offensive rhythm through effective decisions over the four-year window.
For analysts, scouts, and coaching staff, this graph highlights the
players who demonstrate reliable processing and decision quality---key
components of drive sustainability and offensive success.

\begin{center}\rule{0.5\linewidth}{0.5pt}\end{center}

\subsection{4.2 - Decision Efficiency
Index}\label{decision-efficiency-index}

\subsection{4.2.1 Question 2 --- Which Quarterbacks frequently make
effective decisions that sustain drives and improve offense
efficiency?}\label{question-2-which-quarterbacks-frequently-make-effective-decisions-that-sustain-drives-and-improve-offense-efficiency}

\begin{center}\includegraphics{qb_decision_efficiency_case_study_files/figure-latex/decision_efficiency_index-1} \end{center}

\subsection{4.2.2 Minimum One Season (\textgreater= 300
Dropbacks)}\label{minimum-one-season-300-dropbacks}

\subsection{4.2.3 This is optional!}\label{this-is-optional}

\begin{center}\includegraphics{qb_decision_efficiency_case_study_files/figure-latex/qb_total_DEI-1} \end{center}

```

\textbf{Summary (Q2):}\\
These bar charts are ideal for displaying \textbf{rankings of a
composite metric}, which makes them perfect for visualizing the Decision
Efficiency Index (DEI). This DEI metric was calculated using the
averages of each underlying component (decision rate, EPA, touchdowns,
turnovers, and tempo---into a single score, and bar charts allow for
quick and intuitive comparison across quarterbacks.

\textbf{Key findings from the graphs:}\\
In the four-season DEI chart, quarterbacks like \textbf{Patrick
Mahomes}, \textbf{Josh Allen}, and \textbf{Tua Tagovailoa} emerge as the
most complete decision-makers, pairing strong decision rates with high
EPA, healthy touchdown production, and controlled turnover profiles over
the full sample. The color fill for average EPA reinforces that the
highest-DEI quarterbacks also tend to drive the most efficient offenses.

\textbf{Optional deep dive:}\\
I was curious about how quarterbacks performed regardless of how many
seasons they played, so I also calculated DEI for any QB with at least
300 dropbacks in a single season.In the minimum-one-season DEI chart,
the rankings highlight high-impact quarterbacks such as \textbf{Brock
Purdy} and \textbf{Jordan Love} who reached the 300+ dropback threshold
and delivered standout overall efficiency even without four full seasons
in the dataset. For these two quarterbacks to reach the Top 10 in this
visual and have had a young career, means that these two quarterbacks
are expected to have a long, fulfilling career.

On the other hand, seeing legacy quarterbacks such as \textbf{Tom Brady}
and \textbf{Aaron Rodgers} appear inside the Top 10 is notable. Their
presence reflects not only sustained performance late in their careers
but also long-term mastery of processing speed, defensive recognition,
and situational awareness. Analyzing how these veterans maintain such
high decision efficiency despite aging curves, roster turnover, and
changing schemes could be an entire case study of its own.

Together, these visualizations directly answer Question 2 by identifying
quarterbacks who combine consistency, efficiency, explosiveness, and
ball security. For front offices and coaches, DEI surfaces the most
complete decision-makers who can sustain drives and elevate offensive
performance, while also distinguishing multi-year anchors from emerging
or peak single-season performers.

\begin{center}\rule{0.5\linewidth}{0.5pt}\end{center}

\subsection{4.3 - Improvement or Decline Over
Time}\label{improvement-or-decline-over-time}

\subsection{4.3.1 Question 3 --- Which Quarterbacks show consistent
improvement or decline in decision efficiency over the four
seasons?}\label{question-3-which-quarterbacks-show-consistent-improvement-or-decline-in-decision-efficiency-over-the-four-seasons}

\begin{center}\includegraphics{qb_decision_efficiency_case_study_files/figure-latex/qb_trend_data_plot-1} \end{center}

\subsection{4.4 - Decision Effectiveness \&
Tempo}\label{decision-effectiveness-tempo}

\subsection{4.4.1 Question 4 --- Is there a relationship between
decision effectiveness and decision tempo that lead to overall drive
success?}\label{question-4-is-there-a-relationship-between-decision-effectiveness-and-decision-tempo-that-lead-to-overall-drive-success}

\begin{center}\includegraphics{qb_decision_efficiency_case_study_files/figure-latex/qb_total_four_seasons_DEI_plot-1} \end{center}

\textbf{Summary (Q4):}~The scatterplot is the optimal choice for
analyzing relationships between two continuous variables, making it
ideal for evaluating how decision effectiveness interacts with decision
tempo. Each point represents a quarterback's combined efficiency and
processing speed profile, while the regression line clarifies the
overall direction of the relationship.

\textbf{Key findings from the graph:}~In this dataset, the fitted
regression line slopes downward, indicating that quarterbacks with
\textbf{higher decision effectiveness(based off of the DEI) tend to have
shorter average time to throw}. Many of the most efficient
decision-makers cluster in the region with both relatively high decision
rates and faster tempos, while less efficient quarterbacks appear more
often in the slower-time-to-throw range. There may be a few
outliers---such as \textbf{Lamar Jackson}, \textbf{Josh Allen}, \&
\textbf{Brock Purdy} who maintains above-average effectiveness despite a
slightly longer time to throw---but the overall pattern supports the
idea that \textbf{quick processing and decisive reads are associated
with better outcomes.}

This visualization directly answers Question 4 by demonstrating a
negative relationship between decision effectiveness and time to throw:
quarterbacks who make faster decisions generally achieve higher
effectiveness. Faster processing reduces pressure exposure, supports
timing-based passing concepts, and maintains offensive rhythm. For
coaches and coordinators, understanding this relationship helps refine
scheme design and identify quarterbacks whose processing speed drives
better offensive results.

\begin{center}\rule{0.5\linewidth}{0.5pt}\end{center}

\section{5. Share}\label{share}

\subsection{5.1 Key Takeaways}\label{key-takeaways}

\begin{itemize}
\item
  \textbf{High decision efficiency correlates with positive outcomes.}\\
  Quarterbacks with higher decision rates and DEI scores more
  consistently sustain drives, avoid negative plays, and generate
  positive EPA.
\item
  \textbf{DEI surfaces the most complete decision-makers.}\\
  Because it blends decision rate, EPA, touchdowns, turnovers, and
  tempo, DEI highlights quarterbacks who balance efficiency, impact, and
  ball security.
\item
  \textbf{Trends highlight long-term risers and fallers.}\\
  Four-year decision efficiency trends reveal which quarterbacks are
  improving, plateauing, or declining, which is critical for long-term
  planning.
\item
  \textbf{Faster tempo generally supports better decision-making.}\\
  The negative relationship between time to throw and decision
  effectiveness suggests that quicker processors tend to produce better
  outcomes.
\end{itemize}

\subsection{5.2 Presenting to
Stakeholders}\label{presenting-to-stakeholders}

This analysis can be communicated and applied across several key
stakeholder groups:

\subsubsection{5.2.1 Player Evaluation \&
Scouting}\label{player-evaluation-scouting}

\begin{itemize}
\tightlist
\item
  Identify quarterbacks with strong multi-year decision profiles.\\
\item
  Highlight emerging players who already exhibit high DEI and decision
  rate.\\
\item
  Flag potential regression risks through declining decision trends.
\end{itemize}

\subsubsection{5.2.2 Coaching \& Scheme
Development}\label{coaching-scheme-development}

\begin{itemize}
\tightlist
\item
  Align offensive schemes with each quarterback's processing strengths
  and tempo.\\
\item
  Use decision efficiency and tempo metrics to refine route concepts,
  timing, and progression structures.\\
\item
  Target coaching interventions for QBs with slower tempo or elevated
  turnover rates.
\end{itemize}

\subsubsection{5.2.3 Analytics \& Front Office
Decision-Making}\label{analytics-front-office-decision-making}

\begin{itemize}
\tightlist
\item
  Incorporate DEI and decision trends into contract, trade, and draft
  models.\\
\item
  Combine DEI with salary and age curves to assess long-term value.\\
\item
  Use decision metrics as an additional lens alongside EPA/play, CPOE,
  and traditional passer rating.
\end{itemize}

\subsubsection{5.2.4 Media, Endorsements \& Narrative
Analysis}\label{media-endorsements-narrative-analysis}

\begin{itemize}
\tightlist
\item
  Support deeper storytelling around which quarterbacks truly drive
  efficient offense.\\
\item
  Use DEI and decision rate to contextualize highlight plays, awards,
  and legacy debates.\\
\item
  Inform endorsement and branding decisions around quarterbacks who pair
  efficiency with stability.
\end{itemize}

\begin{center}\rule{0.5\linewidth}{0.5pt}\end{center}

\section{6. Act}\label{act}

\subsection{6.1 Recommendations}\label{recommendations}

Based on the four-year analysis of quarterback decision-making, tempo,
and efficiency, the following recommendations are proposed for coaching
staff, analysts, and front office decision-makers:

\subsubsection{6.1.1 Integrate Decision Efficiency Index (DEI) Into
Standard QB
Evaluation}\label{integrate-decision-efficiency-index-dei-into-standard-qb-evaluation}

\begin{itemize}
\tightlist
\item
  DEI proved to be a reliable, multi-factor measure of a quarterback's
  ability to sustain drives, avoid negative plays, and generate positive
  EPA.\\
\item
  Teams should incorporate DEI alongside traditional metrics (EPA/play,
  CPOE, passer rating) to gain a fuller picture of QB performance.
\end{itemize}

\subsubsection{6.1.2 Emphasize QBs With Multi-Year Positive
Trends}\label{emphasize-qbs-with-multi-year-positive-trends}

\begin{itemize}
\tightlist
\item
  Quarterbacks who showed continuous improvement from 2021 to 2024 (such
  as Tua Tagovailoa and Patrick Mahomes) demonstrate stable development
  patterns.\\
\item
  These players represent low-risk, high-stability options for long-term
  schemes and system planning.
\end{itemize}

\subsubsection{6.1.3 Use Decision Tempo as a Diagnostic
Tool}\label{use-decision-tempo-as-a-diagnostic-tool}

\begin{itemize}
\tightlist
\item
  Faster average time to throw correlates with higher decision
  effectiveness in this dataset.\\
\item
  Teams should monitor tempo to:

  \begin{itemize}
  \tightlist
  \item
    Identify processing-speed improvements or regressions\\
  \item
    Evaluate QB fit with quick-game vs.~deep-dropback schemes\\
  \item
    Detect pressure-handling issues before they manifest as turnovers
  \end{itemize}
\end{itemize}

\subsubsection{6.1.4 Prioritize Quarterbacks With High Decision Volume
\& High Decision
Quality}\label{prioritize-quarterbacks-with-high-decision-volume-high-decision-quality}

\begin{itemize}
\tightlist
\item
  Dropbacks represent opportunity, but DEI and decision rate capture
  quality.\\
\item
  Quarterbacks who rank highly in both categories are the most
  dependable leaders of sustainable offense.
\end{itemize}

\subsubsection{6.1.5 Apply Findings to Scheme
Design}\label{apply-findings-to-scheme-design}

\begin{itemize}
\tightlist
\item
  Build offenses that reinforce quick, high-confidence decision
  pathways.\\
\item
  Use motion and spread concepts to help slower processors.\\
\item
  Increase timing-based routes for QBs with high tempo and high DEI
  profiles.
\end{itemize}

\subsection{6.2 Next Steps}\label{next-steps}

\subsubsection{6.2.1 Expand the Dataset to 10
Seasons}\label{expand-the-dataset-to-10-seasons}

A decade-long dataset would:

\begin{itemize}
\tightlist
\item
  Reveal deeper efficiency and tempo evolution.\\
\item
  Highlight outliers in long-term QB development.\\
\item
  Strengthen comparisons across eras, systems, and coaching staffs.
\end{itemize}

\subsubsection{6.2.2 Add Situational Decision Efficiency
Metrics}\label{add-situational-decision-efficiency-metrics}

DEI can be extended to:

\begin{itemize}
\tightlist
\item
  3rd-down DEI\\
\item
  Red zone DEI\\
\item
  Two-minute drill DEI\\
\item
  Play-action vs.~straight dropback DEI
\end{itemize}

These situational splits would help identify QBs with specialized
strengths and pressure performance.

\subsubsection{6.2.3 Include Pressure-Based Decision
Metrics}\label{include-pressure-based-decision-metrics}

Integrating Next Gen Stats pressure variables could help answer:

\begin{itemize}
\tightlist
\item
  How does decision tempo change under pressure?\\
\item
  Which QBs maintain high DEI in collapsing pockets?\\
\item
  How do blitz-heavy defenses affect effectiveness?
\end{itemize}

\subsubsection{6.2.4 Explore Drive-Level
Efficiency}\label{explore-drive-level-efficiency}

A future iteration could model:

\begin{itemize}
\tightlist
\item
  Expected points per drive\\
\item
  Drive success rate\\
\item
  Conversion efficiency (first downs gained per opportunity)
\end{itemize}

This would strengthen the connection between QB decision quality and
actual scoring output.

\subsubsection{6.2.5 Extend the Framework Beyond
QBs}\label{extend-the-framework-beyond-qbs}

This DEI framework could be adapted for:

\begin{itemize}
\tightlist
\item
  Running backs (run decision efficiency)\\
\item
  Wide receivers (separation and YAC decisions)\\
\item
  Defensive backs (target decision efficiency)
\end{itemize}

and for future MLB or other sports analytics projects.

\subsubsection{6.2.6 Build a Public Dashboard or Coaching
Report}\label{build-a-public-dashboard-or-coaching-report}

Future deliverables could include:

\begin{itemize}
\tightlist
\item
  A Shiny app for interactive DEI exploration.\\
\item
  A Tableau dashboard for trend and ranking visuals.\\
\item
  A coaching report with player-specific recommendations and decision
  profiles.
\end{itemize}

\end{document}
